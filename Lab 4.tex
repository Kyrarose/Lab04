% Digital Logic Report Template
% Created: 2020-01-10, John Miller

%==========================================================
%=========== Document Setup  ==============================

% Formatting defined by class file
\documentclass[11pt]{article}

% ---- Document formatting ----
\usepackage[margin=1in]{geometry}	% Narrower margins
\usepackage{booktabs}				% Nice formatting of tables
\usepackage{graphicx}				% Ability to include graphics

%\setlength\parindent{0pt}	% Do not indent first line of paragraphs 
\usepackage[parfill]{parskip}		% Line space b/w paragraphs
%	parfill option prevents last line of pgrph from being fully justified

% Parskip package adds too much space around titles, fix with this
\RequirePackage{titlesec}
\titlespacing\section{0pt}{8pt plus 4pt minus 2pt}{3pt plus 2pt minus 2pt}
\titlespacing\subsection{0pt}{4pt plus 4pt minus 2pt}{-2pt plus 2pt minus 2pt}
\titlespacing\subsubsection{0pt}{2pt plus 4pt minus 2pt}{-6pt plus 2pt minus 2pt}

% ---- Hyperlinks ----
\usepackage[colorlinks=true,urlcolor=blue]{hyperref}	% For URL's. Automatically links internal references.

% ---- Code listings ----
\usepackage{listings} 					% Nice code layout and inclusion
\usepackage[usenames,dvipsnames]{xcolor}	% Colors (needs to be defined before using colors)

% Define custom colors for listings
\definecolor{listinggray}{gray}{0.98}		% Listings background color
\definecolor{rulegray}{gray}{0.7}			% Listings rule/frame color

% Style for Verilog
\lstdefinestyle{Verilog}{
	language=Verilog,					% Verilog
	backgroundcolor=\color{listinggray},	% light gray background
	rulecolor=\color{blue}, 			% blue frame lines
	frame=tb,							% lines above & below
	linewidth=\columnwidth, 			% set line width
	basicstyle=\small\ttfamily,	% basic font style that is used for the code	
	breaklines=true, 					% allow breaking across columns/pages
	tabsize=3,							% set tab size
	commentstyle=\color{gray},	% comments in italic 
	stringstyle=\upshape,				% strings are printed in normal font
	showspaces=false,					% don't underscore spaces
}

% How to use: \Verilog[listing_options]{file}
\newcommand{\Verilog}[2][]{%
	\lstinputlisting[style=Verilog,#1]{#2}
}




%======================================================
%=========== Body  ====================================
\begin{document}

\title{ELC 2137 Lab 04: Subtractor}
\author{Kyra Rose}

\maketitle


\section*{Summary}

In this lab we modified our full adder from the last lab, to make the two bit adder/subtractor in this lab. The only adjustment we had to make to our previous adder was to add two additional XOR gates with new inputs and a mode input (previously Cin). 


\section*{Q\&A}

1) Why did we use two full adders instead of a half adder and a full adder?

We used the 2 full adders we built in the last lab instead of a half adder and full adder, because it was a simple modification. 

2) How many input combinations would it take to exhaustively test the adder/subtractor?

4069

3) Why were the combinations given in the truth table chosen?

These values test all the possible combinations in one test. 

4) Do the results from your adder/subtractor match what you would expect from theory? Explain any discrepancies. 

Yes, the results match what I would expect from theory. When you look at the truth table you can see that the last two bits of the subtracted answer match the last two bits of the B inputs 2 compliment. The first bit of the answer is the only different bit, which makes sense because the first bit in the 2 compliment is a sign bit. 


\section*{Results}

\begin{center}
	
	\includegraphics[width=0.5\textwidth]{circuit demo page 4}
	
	\caption{Circuit Demonstration Page}
	
	\includegraphics[width=0.5\textwidth]{schematic}
	
	\caption{Subtractor Wiring Diagram and Schematic}
	
	\includegraphics[width=0.5\textwidth]{subtractor}
	
	\caption{Subtractor}
	
\end{center}


\section*{Code}

\begin{lstlisting}[style=Verilog,
	caption=Direct Verilog code example,
	label=code:ex 
	]
\begin{center}
	
	\includegraphics[width=0.5\textwidth]{circuit demo page 4}
	
	\caption{Circuit Demonstration Page}
	
	\includegraphics[width=0.5\textwidth]{schematic}
	
	\caption{Subtractor Wiring Diagram and Schematic}
	
	\includegraphics[width=0.5\textwidth]{subtractor}
	
	\caption{Subtractor}
	
\end{center}
\end{lstlisting}


\end{document}
